\documentclass{report}
\usepackage{mathtools }
\usepackage{amsmath}
% \usepackage[thinc]{esdiff}
\begin{document}
    \title{Yams}
    \author{SSG Aero S.A.S.}
    \date{Jully 2021}
    \maketitle
    \chapter{Equation set used}
    \section{Original set}

        The equation sets is derived from the set proposed by \cite{Novak77} who proposed an equation for design and an equation for analysis in a three dimensional context.
        Analysis mode:
        \begin{equation}
            \frac{\partial}{\partial q} \left(\frac{1}{2}V_m^2\right) = 
            D\left(q\right) V_m^2 + E\left(q\right) V_m + F\left(q\right)
            \label{eq:analysis_general}
        \end{equation}
        With:
        \begin{equation}
            \begin{multlined}
                D\left(q\right) = \cos ^2 \beta \times \\ \left[
                    \cos \epsilon \cos\left(\gamma+\phi\right)\left(\frac{1}{r_m}+\frac{\tan \beta}{r}\frac{\partial \phi}{\partial \theta}\right)
                    -\frac{\tan \beta}{r}\frac{\partial r \tan \beta}{\partial q}
                    +\frac{\sin \epsilon}{r \cos \beta} \frac{\partial r \tan \beta}{\partial S}
                    \right]
            \end{multlined}
        \end{equation}
        \begin{equation}
            E \left(q\right) = 2 \widetilde{\omega} \cos^2 \beta \left(\sin \epsilon \sin \phi - \cos \gamma \cos \epsilon \tan \beta\right)
        \end{equation}
        \begin{equation}
            \begin{multlined}
                F \left(q\right) = \cos ^2 \beta \left(\frac{\partial I}{\partial q} - t \frac{\partial \overline{S} }{\partial q} \right)
                + \left[
                    \cos \epsilon \cos \beta \sin \left( \gamma + \phi \right) + \sin \epsilon \sin \beta 
                    \right] 
                \\ \times \left[
                    \frac{d}{dS} \left(\frac{1}{2}V_m^2\right)  + \cos^2 \beta \left(f \frac{d\overline{S}}{dS}\right)
                \right]
            \end{multlined}
        \end{equation}
        Design mode:
        \begin{equation}
            \frac{\partial}{\partial q}\left(\frac{1}{2}V_m^2\right) = G \left(q\right) V_m^2 + J \left(q\right) V_m + K \left(q\right)
        \end{equation}
        With:
        \begin{equation}
            G\left(q\right) = \cos \epsilon \cos \left(\gamma + \phi\right) \frac{1}{r_m}+\frac{\tan \beta}{r} \frac{\partial \phi}{\partial \theta}
        \end{equation}
        \begin{equation}
            J \left(q\right) = \frac{\sin \epsilon}{r \cos \beta} \frac{d}{dS}\left(rV_{\theta}\right)
        \end{equation}
        \begin{equation}
            \begin{multlined}
                K \left(q\right) = \frac{\partial H}{\partial q} - t \frac{\partial \overline{S}}{\partial q} - \frac{V_{\theta}}{r} \frac{rV_{\theta}}{\partial q} + 
                \frac{\cos \epsilon \sin \left(\gamma + \phi\right)}{\cos \beta} \frac{\partial \frac{1}{2}V_m^2}{\partial S} \\
                + \left(\cos \epsilon \cos \beta \sin \left(\gamma + \phi \right) +  \sin \epsilon \sin \beta \right) t \frac{\partial \overline{S}}{\partial S}
            \end{multlined}
        \end{equation}
        \(D\left(q\right)\), \(E\left(q\right)\), \(F\left(q\right)\) are supposed independent from \(V_m\). The gradients from \(F\left(q\right)\) is computed from previous iteration.
        \begin{flalign}
            I = H - \widetilde{\omega}V_{\theta} &&
        \end{flalign}
        \begin{flalign} 
            \gamma = \tan^{-1} \left(dz/dr\right) \mbox{On computing station} &&
        \end{flalign}
        \begin{flalign}
            \phi = \tan^{-1} \left(dr/dz\right) \mbox{On stream line} &&
        \end{flalign}
        \section{Simplification for axisymmetric meridional flow}
        The previous set of equation was designed to take into account the tangential variation of parameters and to solve the equation on severals meridional planes. Here with the assumption of meridional flow whe can make some assumptions.
        \begin{flalign}
            \frac{d}{dS} = \cos \beta \frac{d}{dm} &&
        \end{flalign}
        \begin{flalign}
            \frac{\partial}{\partial q} = \cos \epsilon \frac{d}{d\overline{q}} &&
        \end{flalign}
        \begin{flalign}
            \frac{d}{d \theta} = 0 &&
        \end{flalign}

        Analysis mode:
        \begin{equation}
            \frac{\partial}{\partial \overline{q}} \left(\frac{1}{2}V_m^2\right) = 
            \overline{D}\left(\overline{q}\right) V_m^2 + \overline{E}\left(\overline{q}\right) V_m + \overline{F}\left(\overline{q}\right)
            \label{eq:analysis}
        \end{equation}
        With:
        \begin{equation}
            \begin{multlined}
                \overline{D}\left(\overline{q}\right) = \cos ^2 \beta \left[
                   \cos \left(\gamma+\phi\right)\left(\frac{1}{r_m}\right)
                    -\frac{\tan \beta}{r}\frac{\partial r \tan \beta}{\partial \overline{q}}
                    +\frac{\tan \epsilon}{r} \frac{\partial r \tan \beta}{\partial m}
                    \right]
            \end{multlined}
        \end{equation}
        \begin{equation}
            \overline{E} \left(\overline{q}\right) = 2 \widetilde{\omega} \cos^2 \beta \left(\tan \epsilon \sin \phi - \cos \gamma \tan \beta\right)
        \end{equation}
        \begin{equation}
            \begin{multlined}
                \overline{F} \left(\overline{q}\right) = \cos ^2 \beta \left(\frac{\partial I}{\partial \overline{q}} - t \frac{\partial \overline{S} }{\partial \overline{q}} \right)
                + \left[
                    \cos \beta \sin \left( \gamma + \phi \right) + \tan \epsilon \sin \beta 
                    \right] 
                \\ \times \left[
                    \cos \beta \frac{d}{dm} \left(\frac{1}{2}V_m^2\right)  + \cos^3 \beta \left(t \frac{d\overline{S}}{dm}\right)
                \right]
            \end{multlined}
        \end{equation}
        Design mode:
        \begin{equation}
            \frac{\partial}{\partial \overline{q}} \left(\frac{1}{2}V_m^2\right) = \overline{G} \left(\overline{q}\right) V_m^2 + \overline{J} \left(\overline{q}\right) V_m + \overline{K} \left(\overline{q}\right)
        \end{equation}
        With:
        \begin{equation}
            \overline{G}\left(\overline{q}\right) = \cos \left(\gamma + \phi\right) \frac{1}{r_m}
        \end{equation}
        \begin{equation}
            \overline{J} \left(\overline{q}\right) = \frac{\tan \epsilon}{r} \frac{d}{dm}\left(rV_{\theta}\right)
        \end{equation}
        \begin{equation}
            \begin{multlined}
                \overline{K} \left(\overline{q}\right) = \frac{\partial H}{\partial \overline{q}} - t \frac{\partial \overline{S}}{\partial \overline{q}} - \frac{V_{\theta}}{r} \frac{rV_{\theta}}{\partial \overline{q}} + 
                \sin \left(\gamma + \phi\right) \frac{\partial \frac{1}{2}V_m^2}{\partial m} \\
                + \left(\cos^2 \beta \sin \left(\gamma + \phi \right) +  \tan \epsilon \sin \beta \cos \beta \right) t \frac{\partial \overline{S}}{\partial m}
            \end{multlined}
        \end{equation}
    \chapter{Integration of \(V_m\)}
    
    The software use the equation set as defined in \cite{Novak77}
    \bibliographystyle{unsrt}
    \bibliography{yams}
\end{document}